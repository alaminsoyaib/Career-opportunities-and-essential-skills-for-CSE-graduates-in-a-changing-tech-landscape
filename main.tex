%% 
%% Copyright 2007-2020 Elsevier Ltd
%% 
%% This file is part of the 'Elsarticle Bundle'.
%% ---------------------------------------------
%% 
%% It may be distributed under the conditions of the LaTeX Project Public
%% License, either version 1.2 of this license or (at your option) any
%% later version.  The latest version of this license is in
%%    http://www.latex-project.org/lppl.txt
%% and version 1.2 or later is part of all distributions of LaTeX
%% version 1999/12/01 or later.
%% 
%% The list of all files belonging to the 'Elsarticle Bundle' is
%% given in the file `manifest.txt'.
%% 
%% Template article for Elsevier's document class `elsarticle'
%% with harvard style bibliographic references
% test
%\documentclass[preprint,12pt,authoryear]{elsarticle}

%% Use the option review to obtain double line spacing
%% \documentclass[authoryear,preprint,review,12pt]{elsarticle}

%% Use the options 1p,twocolumn; 3p; 3p,twocolumn; 5p; or 5p,twocolumn
%% for a journal layout:
%% \documentclass[final,1p,times,authoryear]{elsarticle}
%% \documentclass[final,1p,times,twocolumn,authoryear]{elsarticle}
%% \documentclass[final,3p,times,authoryear]{elsarticle}
%% \documentclass[final,3p,times,twocolumn,authoryear]{elsarticle}
%% \documentclass[final,5p,times,authoryear]{elsarticle}
 \documentclass[final,5p,times,twocolumn]{elsarticle}

%% For including figures, graphicx.sty has been loaded in
%% elsarticle.cls. If you prefer to use the old commands
%% please give \usepackage{epsfig}

%% The amssymb package provides various useful mathematical symbols
\usepackage{amssymb}
\usepackage{lipsum}
\usepackage{tabularray}
%% The amsthm package provides extended theorem environments
%% \usepackage{amsthm}

%% The lineno packages adds line numbers. Start line numbering with
%% \begin{linenumbers}, end it with \end{linenumbers}. Or switch it on
%% for the whole article with \linenumbers.
%% \usepackage{lineno}

%% You might want to define your own abbreviated commands for common used terms, e.g.:
\newcommand{\kms}{km\,s$^{-1}$}
\newcommand{\msun}{$M_\odot$}

\journal{Physics Open}

\begin{document}

\begin{frontmatter}

%% Title, authors and addresses

%% use the tnoteref command within \title for footnotes;
%% use the tnotetext command for theassociated footnote;
%% use the fnref command within \author or \affiliation for footnotes;
%% use the fntext command for theassociated footnote;
%% use the corref command within \author for corresponding author footnotes;
%% use the cortext command for theassociated footnote;
%% use the ead command for the email address,
%% and the form \ead[url] for the home page:
%% \title{Title\tnoteref{label1}}
%% \tnotetext[label1]{}
%% \author{Name\corref{cor1}\fnref{label2}}
%% \ead{email address}
%% \ead[url]{home page}
%% \fntext[label2]{}
%% \cortext[cor1]{}
%% \affiliation{organization={},
%%            addressline={}, 
%%            city={},
%%            postcode={}, 
%%            state={},
%%            country={}}
%% \fntext[label3]{}

\title{Career opportunities and essential skills for CSE graduates in a changing tech landscape}

%% use optional labels to link authors explicitly to addresses:
%% \author[label1,label2]{}
%% \affiliation[label1]{organization={},
%%             addressline={},
%%             city={},
%%             postcode={},
%%             state={},
%%             country={}}
%%
%% \affiliation[label2]{organization={},
%%             addressline={},
%%             city={},
%%             postcode={},
%%             state={},
%%             country={}}

\author[first]{Md. Al-Amin Hossain Soyaib}
\affiliation[first]{organization={Stamford University Bangladesh},Department of Computer Science and Engineering,
            % addressline={}, 
            city={Dhaka},
            % postcode={}, 
            % state={},
            country={Bangladesh}
            }
            
\author[second]{Sabbir Hossain Rizvi}
\affiliation[second]{organization={Stamford University Bangladesh},Department of Computer Science and Engineering,
            % addressline={}, 
            city={Dhaka},
            % postcode={}, 
            % state={},
            country={Bangladesh}
            }

\author[third]{Saifur Rahaman Mazumder }
\affiliation[third]{organization={Stamford University Bangladesh},Department of Computer Science and Engineering,
            % addressline={}, 
            city={Dhaka},
            % postcode={}, 
            % state={},
            country={Bangladesh}
            }

\begin{abstract}
%% Text of abstract
Example abstract for the Physics Open journal. Here you provide a brief summary of the research and the results.
\end{abstract}

%%Graphical abstract
%\begin{graphicalabstract}
%\includegraphics{grabs}
%\end{graphicalabstract}

%%Research highlights
%\begin{highlights}
%\item Research highlight 1
%\item Research highlight 2
%\end{highlights}

\begin{keyword}
%% keywords here, in the form: keyword \sep keyword, up to a maximum of 6 keywords
keyword 1 \sep keyword 2 \sep keyword 3 \sep keyword 4

%% PACS codes here, in the form: \PACS code \sep code

%% MSC codes here, in the form: \MSC code \sep code
%% or \MSC[2008] code \sep code (2000 is the default)

\end{keyword}


\end{frontmatter}

%\tableofcontents

%% \linenumbers

%% main text

\section{Introduction}
\subsection{Background}

Industries have evolved because of rapid technological advances, which has created demand for skilled CSE professionals. Because technologies like cloud computing, big data analytics, artificial intelligence, and the Internet of Things today support daily living and manufacturing processes, modern technological skills is necessary.  Industry 4.0 and the digital economy are compelling higher education institutions to revise their academic programs to equip graduates with the competencies needed in evolving workplaces \cite{1_MisraKhurana2017} . In addition to fundamental programming skills, data analytics, machine learning, and cybersecurity competencies are now essential for CSE professions. New graduates need to continue learning to remain relevant due to the fact that technology is changing so rapidly \cite{3_DeSilva2024} . In addition to technical skills, employers also highly regard soft skills such as teamwork, communication, problem-solving, and adaptability.
These skills are presently viewed as key drivers of career success on par with the level of sector-specific knowledge for most employers \cite{5_MohammedOzdamli2024} . Even amid strong demand in fields such as Bangladesh's IT sector, where job placement for CSE graduates reaches as much as 77 percent, a gap in skills between education by schools and industry remains. Most graduates do not possess up-to-date, industry-relevant skills or in-job experience that restricts their own likelihood of direct employability without additional training \cite{4_Chakraborty2019} . New global career paths are being opened to individuals who acquire proficiency in emerging fields such as data science and artificial intelligence. As CSE graduates drive innovation and digital transformation in every sector of the global economy, it is crucial to solve these problems \cite{6_PrinceIdrisKawserAlif2025} .

\subsection{Motivation}
In light of IR 4.0, Malaysian researcher Poh Kiong Tee \cite{M1_tee2024bridging} provides an essential perspective on the digital skills required and the talent deficits.The results of the investigation refuted the notion that eagerness to pay for microcredentials has a modulating impact on employability. The present study verified how well microcredentialing fills digital skill gaps; future research will concentrate on other digital skill domains that affect graduate employability. This research confirms how entry-level graduate employees' employability is impacted by microcredentials and digital skills.These findings suggest that educational programs should put development of these competencies first so as to better get pupils ready for the workplace.Employers could also profit from recognizing the need of microcredentials in assessing prospective candidates. Mehrdad Maghsoudi \cite{M2_maghsoudi2024uncovering}, Four different skill groups were found inside the network: Generalist, Infrastructure and Security, Software Development, and Embedded Systems.Generally, the research offers insightful information on the present condition of the computer science job market and can help individuals and companies make wise decisions regarding skills development and Professionals looking for employment or career growth in the computer science field should consider acquiring these highly sought-after skills to boost their employability and job prospects.For the CSE Department students looking for jobs, this paper is so good; but, this work can be more grand since the CSE industry is not constrained by these four clearly defined abilities. Cheng Peiwen \cite{M3_peiwen2025impact} conducted a research where he write AI affects several sectors quite differently, though lowskilled jobs are fast being replaced, the demand for highskilled ones and fresh roles is growing, therefore changing the job market toward a more intelligent one with advanced countries and hightech sectors adjusting fast to this change while less developed areas and conventional industries are under more pressure to transform. The dynamic and complex link between artificial intelligence and employment requires multiple studies to guarantee that technical developments improve the job market.Future studies could assist to assess the efficacy of current rules and to measure the effects of artificial intelligence on the job market in several sectors and locations using empirical analysis.In many ways, artificial intelligence and employment are related; thus, in several methods, these links have to be investigated to guarantee that technological advancements support the job market. Lately, as we wrap up our computer science degree, we've been struck by how quickly the world of work is changing. It feels like every few months there's a new tool or framework, and our classes can't always keep up. Researchers say the digital revolution and Industry 4.0 have turned the job market on its head; employers now want people who can juggle information, collaborate online, design digital content, safeguard data and solve problems \cite{M4_1_tee2024demand}. Sadly, many graduates still fall short on these basics and critics warn that universities are lagging \cite{M4_1_tee2024demand}. We see why: a lot of syllabuses still focus on classic theory while exciting fields like data analysis, AI and simulation barely get airtime. Some academics argue for a broader approach to digital literacy that blends technical skills with critical thinking, teamwork, ethics and creativity. Others admit they don't feel ready to teach these emerging topics or to collaborate across departments \cite{M5_2_huang2025digital}. No wonder reviewers note that graduates often lack both technical and soft digital skills because it's hard to predict exactly what industry will need next \cite{M4_1_tee2024demand}. Reports from the labour market back this up: a World Bank brief states the digital economy is growing six times faster than the rest and could soon represent a quarter of global GDP \cite{M11_8_elzir2024building}. The same brief suggests that 83 million jobs might vanish by 2025 while 69 million new ones emerge, with almost half of all workers needing to update their skills \cite{M11_8_elzir2024building}. It also mentions that digital literacy is basically mandatory now; more than ninety per cent of US job ads ask for it and employers especially value people who can think analytically, dream up new ideas and keep learning. Meanwhile, the International Labour Organization finds that about a quarter of jobs worldwide now involve generative AI in some way, yet it reassures us that most occupations will evolve rather than disappear \cite{M12_10_ilo2025impact}. We're also seeing whole sectors blossom: healthcare, finance, manufacturing, retail and entertainment are all investing heavily in AI \cite{M14_12_nexford2025ai}. Freelancing platforms report a boom in technical gigs like AI modelling and data annotation, but also in people-centric roles such as coaching and training; specialists with deep AI expertise are commanding higher fees \cite{M13_11_upwork2025skills}. As for us, we've mastered C, C++, Java, data structures and some GUI work, yet we've had little chance to tackle interdisciplinary or AI projects, so we know we'll need to build soft skills and get hands-on experience. Picking languages wisely matters too; surveys put Python at the top for its data and machine-learning libraries. Java still holds its own thanks to its stability in big companies, and JavaScript runs much of the web. There's still room for C++ and C\# in high-performance and enterprise systems; TypeScript is catching up as a scalable alternative; and SQL remains indispensable for managing data in the age of AI \cite{M15_13_pluralsight2025languages}. All these threads point to one conclusion: the future will favour people who combine cutting-edge technical know-how with timeless human qualities like communication, flexibility and ethics. Jobs will morph rather than vanish, so we need to stay curious and push for classes that reflect where technology is going. Our hope with this paper is to spark a conversation about how universities, companies and policymakers can work together to equip computer science graduates for a labour market transformed by AI.
% \usepackage{tabularray}


\section{Literature review}
% \subsection{test}
\begin{table*} 
\centering
\scriptsize
\caption{Role wise evidence table}
\begin{tblr}{
  width=\textwidth,
  colspec = {X[1.1,l] X[2] X[2] X[2] X[1,c]},
  colsep=2pt,
  rowsep=1pt,
  row{1} = {font=\bfseries},
  hlines,
}
Role & Future-Proof Growth Factors & Essential Skills for This Role & Typical tools & AI replaceability risk \\
Backend Software Engineer & Every product needs secure and scalable APIs, payments, and integrations across sectors like fintech, health, and logistics. & System design; OOP; data structures; SQL and NoSQL; REST and GraphQL; caching; message queues; testing. & Node.js; Express; Java Spring Boot; Python FastAPI or Django; Go; PostgreSQL; MySQL; MongoDB; Redis; Kafka; Docker; Kubernetes. & Medium \\
Frontend Engineer & Web remains the main user interface for consumer and enterprise apps. Accessibility and performance are critical. & TypeScript; accessibility; performance budgets; state management; testing; design systems. & React or Vue or Angular; Next.js; Tailwind; Vite; Vitest; Playwright; Web Vitals. & Medium to High \\
Mobile App Developer & Continuous growth of mobile usage, super apps, fintech, and on device AI features. & Kotlin or Swift; Flutter or React Native; offline first sync; app architecture; performance; security. & Android Studio; Xcode; Flutter; Firebase; Realm; SQLite. & Medium \\
Data Engineer & Data pipelines are the foundation for BI and AI. Companies move to lakehouse and streaming. & Advanced SQL; Python; data modeling; ETL and ELT; streaming; orchestration; data quality. & Airflow; dbt; Spark; Kafka; Flink; Snowflake; BigQuery; Delta Lake; Great Expectations. & Low to Medium \\
Machine Learning Engineer & AI features are moving from demo to production across many industries. & Python; classic ML; deep learning; feature engineering; evaluation; deployment; vector search. & scikit learn; PyTorch; TensorFlow; Hugging Face; ONNX; MLflow; Weights and Biases. & Low to Medium \\
MLOps Engineer & Organizations need reliable training, deployment, and monitoring of models at scale. & CI/CD for models; feature stores; data and model versioning; monitoring; drift detection. & MLflow; DVC; Feast; Kubeflow; Seldon; Vertex AI; Docker; Kubernetes. & Low \\
Data Scientist & Product analytics and experimentation drive growth and decisions. & Statistics; causal inference; A/B testing; SQL; Python; visualization; storytelling. & pandas; NumPy; SciPy; scikit learn; Jupyter; Tableau or Power BI. & Medium \\
Cybersecurity Analyst or Engineer & Rising threats, compliance rules, and AI powered attacks increase demand. & Threat modeling; secure coding; network analysis; incident response; red and blue teaming. & Wireshark; Zeek; Suricata; SIEM like Splunk; Burp Suite; Metasploit; OSINT tools. & Low \\
Cloud and DevOps Engineer & Cloud adoption continues with cost and reliability focus. & Infrastructure as code; CI/CD; container orchestration; observability; FinOps. & AWS or Azure or GCP; Terraform; Ansible; Docker; Kubernetes; Prometheus; Grafana. & Low to Medium \\
Site Reliability Engineer & Always on digital services need reliability and fast incident response. & SLI and SLO design; error budgets; incident management; capacity planning; observability. & Prometheus; Grafana; OpenTelemetry; Sentry; PagerDuty; Chaos tools. & Low \\
Blockchain or Smart Contract Engineer & Enterprise tokenization, DeFi, and digital identity remain active niches. & Solidity or Rust; contract security; cryptography basics; formal verification. & Hardhat; Foundry; OpenZeppelin; Substrate; Ethers.js; Web3.js. & Low \\
AR or VR or XR Developer & Enterprise training, retail try ons, education, and gaming push immersive tech. & Unity or Unreal; C\#; 3D math; shaders; interaction design; performance. & Unity; Unreal; Blender; ARKit; ARCore; OpenXR. & Medium \\
IoT Systems Engineer & Smart industry, agriculture, and cities need secure low power devices and edge AI. & Embedded C or C++; microcontrollers; RTOS; MQTT; edge inference; OTA updates. & ESP32; Arduino; Zephyr; FreeRTOS; AWS IoT; Azure IoT; Node RED. & Low \\
QA Automation Engineer & Faster release cycles need strong automated testing and quality gates. & Test strategy; e2e and unit tests; property based tests; performance and security testing. & Selenium; Playwright; Cypress; JUnit; PyTest; Postman; JMeter; k6. & Medium to High \\
GenAI Application Engineer & Teams add language model features for search, support, and automation. & LLM prompt design; RAG; evaluation; vector search; security; cost control. & OpenAI or local LLMs; Hugging Face; LangChain or LlamaIndex; FAISS or Pinecone or Weaviate; Guardrails. & Medium 
\end{tblr}

\bigskip
\noindent\begin{minipage}{\textwidth}
\footnotesize
The table presents a role wise view of career paths across core domains of computer science and engineering, summarizing growth drivers, essential skills, common tooling, and the relative risk of replacement by AI. It spans fifteen roles that cover software product work, data and AI, infrastructure and reliability, security, and emerging areas such as blockchain, spatial computing, and the internet of things. Growth factors emphasize durable demand for secure and scalable backends, the central place of the web and mobile as user touchpoints, the rise of data pipelines for business intelligence and AI, the move of machine learning from demos to production, and the continued expansion of cloud native platforms with strict reliability and compliance needs. The skills column maps each role to concrete competencies, from system design, SQL, and algorithms to TypeScript and accessibility, from Kotlin or Swift to Python for classic and deep learning, from ETL and streaming to threat modeling, infrastructure as code, observability, and test strategy with end to end coverage. Tooling examples ground these skills in real workflows, listing stacks like Node and Spring for backend, React and Next for frontend, Android Studio and Xcode for mobile, Airflow and dbt for data engineering, PyTorch and TensorFlow for ML, MLflow and Kubeflow for MLOps, SIEM and IDS suites for security, Terraform and Docker for cloud and DevOps, and Prometheus and Grafana for site reliability. Emerging domains are also specified with practical toolchains such as Solidity with Hardhat for smart contracts, Unity or Unreal for AR and VR, microcontroller stacks for embedded IoT, and modern LLM frameworks for retrieval augmented generation and evaluation. The AI replaceability risk column differentiates roles where human judgment, safety, and systems thinking remain central, which tend to show low to medium risk, from areas with more template driven interfaces or repetitive validation, which trend toward medium to high risk. Together these dimensions indicate that future proof careers blend strong fundamentals with platform specific mastery and an ability to ship secure, observable, and maintainable systems. The table also implies that cross cutting skills like version control, collaborative design, performance and cost awareness, and clear communication are valuable across sectors. For students and graduates, the matrix can guide course selection, project focus, and internship targeting by aligning interests with roles that show resilient demand. For educators and mentors, it offers a compact map of where the market is moving and which tools best represent job ready proficiency. Overall, the table frames the changing tech landscape not as a single trend but as a connected set of sectors that need engineers, scientists, and operators who can turn ideas into reliable value.
\end{minipage}
\end{table*} 




% \begin{figure}
% 	\centering 
% 	\includegraphics[width=0.4\textwidth, angle=-90]{PHYSO_cover_image.pdf}	
% 	\caption{Physics Open journal cover} 
% 	\label{fig_mom0}%
% \end{figure}

% \label{introduction}

% Here is where you provide an introduction to work and some background. For example building on previous work of image enhancement in optical astronomy \citep{vojtekova2021learning}, \cite{sweere2022deep} developed a method to improve the resolution of X-ray images from XMM-Newton to provide similar spatial resolution to Chandra.

\section{Title 2}
%%\label{}
\lipsum[1]

\subsection{Subsection title}

A random equation, the Toomre stability criterion:

\begin{equation}
    Q = \frac{\sigma_v \times \kappa}{\pi \times G \times \Sigma}
\end{equation}

\section{Title 3}
%%\label{}
\lipsum[2]

\subsection{Subsection title}
\lipsum[3]

\begin{table}
\begin{tabular}{l c c c} 
 \hline
 Source & RA (J2000) & DEC (J2000) & $V_{\rm sys}$ \\ 
        & [h,m,s]    & [o,','']    & \kms          \\
 \hline
 NGC\,253 & 	00:47:33.120 & -25:17:17.59 & $235 \pm 1$ \\ 
 M\,82 & 09:55:52.725, & +69:40:45.78 & $269 \pm 2$ 	 \\ 
 \hline
\end{tabular}
\caption{Random table with galaxies coordinates and velocities, Number the tables consecutively in
accordance with their appearance in the text and place any table notes below the table body. Please avoid using vertical rules and shading in table cells.
}
\label{Table1}
\end{table}


\section{Discussion}
%%\label{}
\lipsum[4]

\section{Summary and conclusions}
%%\label{}
\lipsum[1-4]


\section*{Acknowledgements}
Thanks to ...

%% The Appendices part is started with the command \appendix;
%% appendix sections are then done as normal sections
\appendix

\section{Appendix title 1}
%% \label{}

\section{Appendix title 2}
%% \label{}

%% If you have bibdatabase file and want bibtex to generate the
%% bibitems, please use
%%
\bibliographystyle{elsarticle-num}
% \bibliographystyle{elsarticle-harv}
% \bibliographystyle{apalike}
% \bibliographystyle{IEEEtran} 
% \bibliographystyle{ieeetr}
\bibliography{example}

%% else use the following coding to input the bibitems directly in the
%% TeX file.

%%\begin{thebibliography}{00}

%% \bibitem[Author(year)]{label}
%% For example:

%% \bibitem[Aladro et al.(2015)]{Aladro15} Aladro, R., Martín, S., Riquelme, D., et al. 2015, \aas, 579, A101


%%\end{thebibliography}

\end{document}

\endinput
%%
%% End of file `elsarticle-template-harv.tex'.
